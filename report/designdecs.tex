\section{Design decisions}
During the design of the project, several design decisions were made, they are listed here.
\begin{itemize}
\item Petrol cards
\begin{itemize}
\item PIN code is used for authenticating the card owner to the terminals.
\item Public Key Infrastructure is used for authentication, encryption and signatures of messages
\item The petrol card will have an initial balance of zero once a car owner receives the petrol card. 
\item All communications between petrol card and terminal is signed \& MACed, except for the first communication of the certificate by either one party.
\item Symmetric crypto is used for encrypted communication between the petrol card and the terminal
\end{itemize}

\item Personalisation terminal
\begin{itemize}
\item Petrol cards will always be given an unique identification number
\item Balance will always be zero 
\item PIN codes will be set by the personalisation terminal
\end{itemize}

\item Charging terminal
\begin{itemize}
\item Charging terminal will relay the signed allowance by the back end to the petrol card 
\item All communications between back-end and terminal is signed \& MACed, except for the first communication of the certificate by either one party.
\item Charging can only be done once a month and only the whole allowance in one time. Allowance cannot be charged in parts.
\item The charging terminal can see the petrol balance stored on the petrol card after the charging terminal has authenticated itself to the petrol card.
\end{itemize}

\item Petrol terminal
\begin{itemize}
%\item Will ask for amount of fuel that is to be released. Card holder knows how much fuel is going to be used. Once entered, the card holder does not get rest of allowance back. So the pump terminal is not able to write allowance to the card.
\item The petrol card will have to remain in the petrol terminal during the whole transaction. After mutual authentication has taken place, the petrol terminal will write a balance of zero to the petrol card, only once the transaction has been finished will the petrol terminal write the new petrol balance (where new balance = old balance - used balance) back to the petrol card. If the pump reaches the maximum amount of balance on the petrol card, the transaction will be ended and no additional petrol can be gained from the petrol pump. 


%will the correct amount of used petrol Card owner will have to insert the petrol card into the petrol terminal for the whole transaction. The petrol terminal will receive the current petrol allowance on the card, the flow of petrol will be terminated if it reaches this number. If the petrol allowance is removed before the transaction is complete, the full petrol allowance will be removed upon the next presentation at a charging or petrol terminal. 

% until the fuel amount is chosen by the card owner. Petrol allowance on the card is provided to terminal, if the card is removed or this amount has been reached, the flow of petrol will be terminated. If the petrol card is removed during this moment, all petrol allowance will be removed from the card ???
%petrol allowance on the card will be decreased before its added to petrol terminal

\end{itemize}


\item Back-end
\begin{itemize}
	\item Will have a signed list of blocked petrol cards that are issued to the terminals.
	\item Will have a list of all ID numbers belonging to each petrol card and terminal
	\item Will have a sub-CA certificate, the utmost level in the certificate chain apart from the main CA.
	\item The main CA certificate will only be used to sign and revoke sub-CA certificates
\end{itemize}
\end{itemize}
